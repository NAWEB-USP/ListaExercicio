\documentclass[12pt,letterpaper]{article}



% Page margin settings
\setlength{\textwidth}{6.5in} \setlength{\textheight}{9.0in}
\setlength{\topmargin}{-.25in} \setlength{\oddsidemargin}{0.0in}
\setlength{\evensidemargin}{0.0in}


% For encapsulated postscript figures
%\usepackage{epsfig}
% For figures in other image formats
\usepackage{graphicx,psfrag}
\usepackage{indentfirst}
\usepackage[brazil]{babel}
\usepackage[utf8]{inputenc}
\usepackage{hyperref}
%\usepackage[a4paper, tmargin=2.5cm, bmargin=2.5cm, lmargin=2.5cm, rmargin=2.5cm]{geometry}


\begin{document}

\sloppy

\title{
{\Large Instituto de Matemática e Estatística} \\
{\large Universidade de S\~ao Paulo, Brasil} \\
\vspace{2cm}
{\bf Academic Devoir}
}


\author{
André Satoshi\\
António Castro\\
Bruno Padilha\\
Gustavo Coelho\\
Luciana Kayo\\
Susanna Rezende\\
Suzana Santos\\
Vinicius Rezende\\
\vspace{2cm}
Wallace Almeida\\
{\small Professor: Marco Gerosa}\\
{\small Disciplina: MAC0332 Engenharia de Software}
\vspace{2cm}
{\small Versão: 2.0}
}

\date{\today}

\maketitle

\thispagestyle{empty}

%\begin{abstract}
%\end{abstract}

\pagebreak

\tableofcontents


%%%%%%%%%%%%%%%%%%%%%%%%%%%%%%%%%%%%%


%\pagebreak
%\section{Introdução}


\pagebreak
\section{Analista}

\subsection{Visão}
``Academic Devoir'' é um sistema a ser desenvolvido para uso em plataforma Web,  utilizando Java como linguagem básica de desenvolvimento, e adotando como tecnologias au\-xi\-liares os frameworks vRaptor e Hibernate.

O sofware tem como característica principal a gestão de tarefas acadêmicas das turmas de variadas disciplinas, baseadas em listas de exercícios, por sua vez constituídas de questões em diversas categorias, como dissertativa e ``Verdadeiro ou Falso'', e de correção automática.

O uso do sistema será destinado a professores e alunos. Os primeiros podem elaborar as atividades e visualizar a correção automática efetuada. Já os últimos poderão exibir as atividades disponíveis e solucioná-las.

\subsection{Requerimentos do Sistema (System-Wide Requirements)}
\subsection{Casos de Uso}
De acordo com o tipo de usuário (aluno ou professor), existirão tarefas que devem ser atendidas pelo sistema. Podemos listar algumas das possiveis utilizações:
\begin{itemize}
\item {Professor gerencia alunos}
\item {Professor cadastra disciplina}
\item {Professor cadastra turma}
\item {Professor cadastra lista de exercício}
\item {Professor cadastra questões}
\item {Professor lista respostas de questão}
\item {Professor pode reordenar questões de uma lista}
\item {Professor determina o tipo de matricula}
\item {Aluno se cadastra}
\item {Aluno se matricula em uma turma de uma disciplina}
\item {Aluno entra em uma disciplina e visualiza as listas}
\item {Aluno resolve uma lista de exercícios}
\end{itemize}

\subsection{Modelo de Casos de Uso}
\subsection{Glossário}


\pagebreak
\section{Gerente}

\subsection{Plano da Iteração}

\vspace{1cm}
{\large {\bf Objetivos da iteração}}
\vspace{0.5cm}


A primeira iteração do projeto tem por objetivo a adaptação dos integrantes aos respectivos papéis e às práticas de desenvolvimento adotadas.

Após as 3 primeiras semanas de desenvolvimento, o grupo deverá apresentar um sistema funcional que realize o cadastro de alunos e professores e o login de usuários.

Não será exigido que o sistema permita o cadastro de disciplinas, turmas, questões ou listas de exercícios, porém, essas funcionalidades deverão estar trabalhando corretamente nos testes  de unidade.

Ao longo da iteração, espera-se que todas as partes do desenvolvimento estejam bem documentadas, conforme a metodologia OpenUP e que seja feita uma boa modelagem das classes do sistema e um levantamento de requisitos detalhado.

\vspace{1cm}
{\large {\bf Atribuições de trabalho}}
\vspace{0.5cm}

As tarefas foram atribuídas conforme o papel dos integrantes. Dependendo da dificuldade da tarefa e de sua relevância para o desenvolvimento do projeto, todos os integrantes devem contribuir.

Segue uma lista dos participantes e de seus respectivos itens de trabalho:

\begin{itemize}
\item {} André Satoshi Fujii de Siqueira (Desenvolvedor)
Divisão das tarefas de implementação
Implementação do CRUD de questões e listas de exercício.
\item {}Antonio Junior (Documentrelacionadaador)
Estudar documentação OpenUP e instruir os integrantes do grupo sobre como deve ser a documentação
Acompanhar a documentação do projeto e dar as orientações necessárias.
Reunir a documetação de todas as partes do desenvolvimento.
\item {}Bruno Padilha (Desenvolvedor)
Divisão das tarefas de implementação
Implementação do cadastro de uma lista de exercícios.
Gustavo Coelho (Analista de qualidade)
Testes de Unidade para os CRUDS desenvolvidos.
Implementação do login no sistema
\item {}Luciana Kayo (Arquiteta)
Modelagem das classes do sistema.
Formatação das páginas com css.
\item {}Susanna Rezende (Arquiteta)
Durante essa iteração, a integrante não terá disponibilidade para executar as tarefas.
O tempo de ausência poderá ser compensado em outras iterações.
\item {}Suzana de S. Santos (Gerente)
Monitoramento das tarefas desenvolvidas e atribuição de novas tarefas
Modelagem das classes do sistema
Implementação do cadastro de uma resposta dada por um aluno no sistema.
\item {}Vinicius Rezende (Desenvolvedor)
Implementação do CRUD de aluno, professor, disciplina e turma.
\item {}Wallace Faveron de Almeida (Analista de requisitos)
Levantamento de requisitos
Implementação do Login no sistema.
\end{itemize}

Para maior detalhamento dos itens de trabalho, visite a página: 
\url{https://www.pivotaltracker.com/projects/355453#}

\vspace{1cm}
{\large {\bf Critérios de avaliação}}
\vspace{0.5cm}

Para avaliarmos o desenvolvimento, usaremos os testes de Unidade e a avaliação do cliente sobre o sistema.
Veja maiores detalhes sobre os testes na documentação do analista de qualidade.


\subsection{Plano do Projeto}


\vspace{1cm}
{\large {\bf Organização}}
\vspace{0.5cm}

O trabalho no projeto é dividido nas seguintes áreas:


\begin{table}[ht!]
\begin{small} %tiny
    \hspace*{-4mm}
    \begin{tabular}{| l | p{5cm} | p{5cm} |}
    \hline
    Identificação & Responsabilidades & Stakeholders\\
    \hline
    \hline
    Gerente do Projeto &
    Atribuições de caráter decisório e estratégico quanto aos rumos do projeto. &
    Suzana Santos\\
    \hline
    Analistas &
    Definir e aprovar os requisitos e especificações de negócio do sistema. &
    Wallace Almeida\\
    \hline
    Arquiteto do Projeto &
    Definir a arquitetura a ser utilizada no sistema. &
    Luciana e Susanna Rezende \\
    & & Colaboradora: Suzana Santos\\
    \hline
    Documentação &
    Documentar &
    António Castro\\
    & & Colaboradores: Todos\\
    \hline
    Programadores &
    Implementar o sistema conforme as especificações. &
    André, Bruno e Vinícius \\
    & & Colaboradores: Todos\\
    \hline
    Testes &
    Padronizar os testes, homologar. &
    Gustavo\\
    \hline
    \end{tabular}
\end{small}
\end{table}


\vspace{1cm}
{\large {\bf Medições}}
\vspace{0.5cm}

A cada item de trabalho atribuímos pontos, que representam horas de dedicação (1 ponto equivale a uma hora).
Na página \url{https://www.pivotaltracker.com/projects/355453#}, temos os seguintes agrupamentos de itens de trabalho:
\begin{enumerate}
\item{Current (trabalho em desenvolvimento, concluído ou a ser desenvolvido em breve)}
\item{Backlog (trabalho a ser desenvolvido em breve)}
\item{Icebox (trabalho a ser desenvolvido sem data para início)}
\end{enumerate}

\vspace{1cm}
{\large {\bf Objetivos}}
\vspace{0.5cm}

Criar um sistema de resolução e correção de listas de exercício na Web.
O desenvolvimento consistirá de 3 iterações (cada uma com 3 semanas de duração) durante as quais serão implementadas as histórias:

\begin{itemize}
\item{Aluno se cadastra}
\item{Login de aluno/professor}
\item{Professor gerencia alunos (CRUD)}
\item{Professor cadastra disciplina}
\item{Professor cadastra turma}
\item{Professor cadastra lista de exercício}
\item{Professor cadastra questão de texto}
\item{Professor cadastra questão de V ou F}
\item{Professor cadastra questão de múltipla escolha}
\item{Professor cadastra questão de submissão de arquivo}
\item{Aluno se matricula em uma turma de uma disciplina}
\item{Professor lista respostas de questão de texto que ainda não foram corrigidas e pode corrigi-la}
\item{Professor pode reordenar questões de uma lista}
\item{Aluno entra em uma disciplina e visualiza as listas a serem feitas e as já corrigidas}
\item{Aluno resolve uma lista de exercícios}
\item{Professor determina o tipo de matricula em uma turma (imediato ou com prazo definido)}
\end{itemize}
Aguardamos mais informações do professor para completar a lista de histórias a serem implementadas.



\subsection{Lista de Riscos}

Perguntar para o professor o que seriam os riscos do projeto.

\subsection{Avaliação de status}

Aguardando feedback do cliente.



\subsection{Lista de Itens de Trabalhos}

A lista dos itens de trabalho pode ser visualizada em: \url{https://www.pivotaltracker.com/projects/355453#}


\pagebreak
\section{Arquiteto}

\subsection{Caderno de Arquitetura}




\pagebreak
\section{Desenvolvedor}

\subsection{Build}
\subsection{Design}
\subsection{Developer Test}
\subsection{Implementation}





\pagebreak
\section{Testador}

A maioria dos testes realizados até o momento são testes nas classes Controller, ou seja, que testam se os Controllers chamam os métodos apropriados dos DAOs e se o usuário é redirecionado para as páginas certas. 

Há também testes que consistem em verificar se os diferentes objetos (Alunos, Professores, etc...) estão sendo corretamente inseridos em suas tabelas no banco de dados.

Para não afetar o banco de dados, todas os objetos que lidam com o banco de dados diratemente são mocks (imitações) dos objetos reais.

\subsection{Test Case}
\subsection{Test Log}
\subsection{Test Script}




%\pagebreak
%\section{Bibliografia}\label{bibliografia}
%\bibliographystyle{plain}
%\bibliography{referencias}   % .bib


\end{document}
